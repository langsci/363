\chapter{Einführung}
\label{chap:1}
Die Markierung der Pluralinformation am Substantiv weist in den Dialekten des Deutschen große formale Variation auf. So finden sich für das mask. Substantiv \textit{Hund}, das im Standarddeutschen\footnote{Wenn im Folgenden der Terminus Standard verwendet wird, dann bezieht sich dies auf die nhd. Schriftsprache als eine Varietät des Deutschen neben anderen.} den Plural additiv mit Schwa-Suffix bildet (\textit{Hund} -- \textit{Hunde}), im Ostfränkischen (Ofr.) Umlautplurale (\teuthoo{hund}{hund} -- \teuthoo{hind}{hind}), Plurale mit Umlaut und einem Kontrast der Vokalquantität (\teuthoo{hu2nd}{hūnd} -- \teuthoo{hu?nd}{hünd}), Nullplurale (\teuthoo{hund}{hund} -- \teuthoo{hund}{hund}), additive Plurale mit Nasalsuffix (\teuthoo{hund}{hund} -- \teuthoo{hundn}{hundn}) oder Schwa-Suffix (\teuthoo{hund}{hund} -- \teuthoo{hundE}{hundə}) und -- im ofr.-hess. Übergangsgebiet -- subtraktive Plurale (\teuthoo{ho2nd}{hōnd} -- \teuthoo{ho?n}{hön}).\footnote{Die Notation von Beispielen in der Form \textit{Hund} -- \textit{Hunde} bezieht sich auf die Formen Nom.Sg. -- Nom.Pl., ohne dass dies im Folgenden explizit gekennzeichnet wird. Bezieht sich das Dialektbeispiel auf eine vom Nom.Sg./Pl. abweichende Flexionsform oder auf eine Wortbildung, wird dies gesondert ausgewiesen, die standardsprachliche Übersetzung folgt direkt in einfachen Anführungszeichen, z.\,B. \teuthoo{hu2nd}{hūnd} -- \teuthoo{hund}{hund} -- Dim. \teuthoo{hu?ndlA}{hündlα} ‚Hund‘. Sprachliche Beispiele erscheinen im Folgenden kursiv, nur im Falle der Teuthonista-Transkripte wurde i.\,d.\,R. darauf verzichtet, um die Lesbarkeit der Diakritika zu gewährleisten. Die Transkriptionen von Dialektbeispielen sind dabei phonetisch zu lesen, ein Lenisplosiv im absoluten Auslaut etwa wurde auch als solcher realisiert. Wenn in den folgenden Ausführungen außerdem zu lesen ist, dass sich nicht-agensfähige Objekte (wie Substantive oder Dialekte) entwickeln oder den Plural markieren, so ist dies metaphorisch zu verstehen.} Daneben sind im Bairischen (Bair.) Pluralformen mit einer Alternation von Lenisplosiv im Stammauslaut des Singulars und Fortisplosiv im Plural (\teuthoo{hund}{hund} -- \teuthoo{hunt}{hunt}) und Plurale mit einer Kombination aus Vokalquantitätskontrast und Lenis-Fortis-Kontrast (\teuthoo{hu2nd}{hūnd} -- \teuthoo{hunt}{hunt}) belegt. All diese Varianten haben einen gemeinsamen Ursprung, da sich die rezenten Dialekte auf Basis des Deklinationssystems der ahd. und mhd. Dialekte entwickelt haben. Es stellt sich daher die Frage: Inwiefern sind die unterschiedlichen Pluralformen lautgesetzliche Fortsetzungen des mhd. Deklinationssystems oder hat sich ein spezifisch dialektales Deklinationssystem entwickelt?

\textit{Hund} gehört historisch zur mask. \textit{a}{}-Deklination, die den Plural im Althochdeutschen additiv mit \textit{a}{}-Suffix (ahd. Nom.Pl. \textit{hunda}) und -- in Folge der Reduzierung der Vollvokale in den Endsilben -- im Mittelhochdeutschen mit Schwa-Suffix (mhd. Nom.Pl. \textit{hunde}) bildet. Sowohl das Ofr. als auch das Bair. sind apokopierende Dialekte, d.\,h. im Laufe des Mittelhochdeutschen entfällt das Schwa-Suffix der Pluralform. Der Nullplural \teuthoo{hund}{hund} -- \teuthoo{hund}{hund} ist damit die Entsprechung der historischen Deklination unter Einwirkung eines phonologischen Prozesses, der Apokopierung des Pluralmarkers. Basisdialektale Pluralformen des Typs \teuthoo{hu2nd}{hūnd} -- \teuthoo{hunt}{hunt} stellen ebenfalls die lautgesetzlichen Fortsetzungen der mhd. Deklination dar, da hier lautgesetzliche Alternationen konserviert sind, die vor der Apokope zwischen einsilbiger Singular- und zweisilbiger Pluralform auftraten: Alternationen zwischen Lang- und Kurzvokal in Folge von Einsilberdehnung und das im Bair. korrelierende Merkmal der Lenis-Fortis-Konsonanz in Abhängigkeit von der Stammvokallänge.

Rezente Pluralformen mit Umlaut oder Nasalsuffix markieren indes einen diachronen Wechsel des Pluralmarkierungsverfahrens und damit der Deklinationsklasse. Auch die basisdialektale Pluralform \teuthoo{hund}{hund} -- \teuthoo{hundE}{hundə} entspricht einem diachronen Wechsel der Deklination, der -- in Kombination mit einem weiteren phonologischen Prozess -- zu einer Pluralform mit Schwa-Suffix auch in einem apokopierenden Dialekt wie dem Ofr. führt: In einem Teil der deutschen Dialekte wird finales /n/ elidiert, aus der Pluralform mit Nasalsuffix *hundən\footnote{Asterisk steht im Folgenden für rekonstruierte Formen.} wird ofr. hundə, die zwar formal der standardsprachlichen Pluralform entspricht, aber das Ergebnis dialektspezifischer phonologischer und morphologischer Prozesse ist. Und schließlich können auch Nullplurale morphologisch bedingt sein, wenn etwa innerparadigmatische Vokalquantitätskontraste ausgeglichen werden.

Anhand dieses einen Beispiels wird deutlich, dass areale Variation in der Pluralmarkierung der rezenten Dialekte das Ergebnis diachroner Entwicklungen auf der Ebene der Phonologie, der Morphologie und an der Schnittstelle von Phonologie und Morphologie ist. Aufgabe einer dialektalen Flexionsmorphologie ist es nun, zu zeigen, wo die formale Varianz der Kodierungsverfahren phonologisch bedingt ist, wo sie das Ergebnis genuin morphologischer Prozesse ist und wo beide Ebenen interagieren. Das Ziel der Untersuchung besteht dabei nicht nur in einer Inventarisierung der einzelnen (evtl. dialektspezifischen) Markierungsstrategien und der arealen Varianten, sondern in der Erfassung des Systems: Inwiefern steuern das phonologische System eines Ortsdialekts und diachrone phonologische Prozesse das zur Verfügung stehende Inventar flexivischer Marker? Welche phonologisch bedingten Alternationen sind -- wie etwa der Umlaut -- morphologisch funktionalisiert und produktiv geworden? Diese Fragen sollen für einen größeren, in sich geschlossenen Dialektraum, das Ostoberdeutsche (Oobd.), beantwortet werden. Da die Datenlage durch den \textit{Bayerischen Sprachatlas} (BSA), his\-to\-ri\-sche und jüngere Dialektgrammatiken und -wörterbücher im Bundesland Bayern besonders gut ist, wird das Untersuchungsgebiet auf die oobd. Dialekte Bayerns eingegrenzt. Die Idee ist dabei, die Spezifika und Gemeinsamkeiten der nominalen Flexionsmorphologie der drei Teilräume des Oobd., Ofr., Nord- und Mittelbair., durch ein hinsichtlich der phonologischen Voraussetzungen möglichst differentes Ortsnetz auszuarbeiten.

Dass dialektale Flexionsmorphologie nicht nur an der Schnittstelle von Phonologie und Morphologie, sondern -- synchron wie diachron -- auch an der Schnittstelle von Morphologie, Syntax und semantisch-pragmatischem Kontext stattfindet, zeigt ein weiteres Beispiel aus den oobd. Dialekten. Bei zweisilbigen Feminina der historisch schwachen Deklination ist das Nasalsuffix der obliquen Kasus analog in den Nom.Sg. übertragen, infolge dieses innerparadigmatischen Ausgleichs erscheinen synkretische Formen, z.\,B. ofr. \teuthoo{glogN}{ɡloɡŋ} -- \teuthoo{glogN}{ɡloɡŋ} ‚Glocke‘, \teuthoo{vlas\#n}{vlašn} -- \teuthoo{vlas\#n}{vlašn} ‚Flasche‘. Die Pluralinformation wird erst im morphosyntaktischen Kontext eindeutig markiert, wobei der fem. Definitartikel -- anders als der Definitartikel bei Maskulina und Neutra --im Nominativ und im Akkusativ nicht dis\-ambiguierend wirkt (Nom./Akk. \textit{die} -- \textit{die}). Während die Numerusinformation von fem. Nominalphrasen in der Subjektposition eindeutig durch die finite Verbform markiert wird (\textit{Die Flaschen ist/sind voll}), bleibt sie bei Akkusativobjekten ambig: \textit{Gib mir die Flaschen} ‚Sg./Pl.‘. Dieses Dilemma uneindeutiger flexivischer Information, das überhaupt erst durch morphologischen Ausgleich entstanden ist, wird in Teilen des Bair. durch eine additive Markierung der Pluralform gelöst: bair. \teuthoo{vlaS'n}{vlaʃ̌n} -- \teuthoo{vlaS'nA}{vlaʃ̌nα}. Diese formale Markierung am Substantiv erfolgt in Teilen des Nord- und Mittelbair. regelmäßig bei Feminina mit Nasalsuffix im Nom.Sg., teilweise ist sie aber auch als fakultative Markierung funktionalisiert: Immer dann, wenn der semantisch-pragmatische Kontext keine eindeutige Unterscheidung bietet, wird die Numerusinformation formal am Substantiv markiert, z.\,B. bair. \teuthoo{naee}{naee} \teuthoo{glognA}{ɡloɡnα} ‚neue Glocken‘, aber -- mit unbestimmtem Zahlwort -- \teuthoo{mid}{mid} \teuthoo{ale}{ale} \teuthoo{glogn}{ɡloɡn} \teuthoo{le2dn}{lēdn} ‚mit allen Glocken läuten‘.

Um diese und ähnliche Formen von Flexion an der Schnittstelle von Morphologie und Syntax und unter Berücksichtigung des semantisch-pragmatischen Kontexts abbilden zu können, umfasst der Untersuchungsgegenstand der vorliegenden Arbeit nicht nur das isolierte Substantiv, sondern die syntaktische Einheit der Nominalphrase bestehend aus Definitartikel und Substantiv. Untersucht werden Numerus- und Kasusflexion von Substantiv und Definitartikel, daneben werden die dialektalen Deklinationsklassen in ihrer synchronen Distribution und Zusammensetzung sowie Deklinationsklassenwandel als diachrones Phänomen analysiert. Damit setzt die Untersuchung an jenen Forschungsaufgaben an, die auch \citet[35]{SchmidtEtAl2019} beschreiben:\pagebreak

\begin{quote}
In der Flexionsmorphologie fehlt es an Untersuchungen, die Systemzusammenhänge und deren Raumbildung im Vergleich mehrerer Varietäten oder Sprachlagen erschließen. Auf paradigmatischer Ebene sind hier z.\,B. die (Re)Organisation von Flexionsklassen bzw. Pluralallomorphie, Kasussysteme oder die Bedingungen von Synkretismen anzuführen.
\end{quote}

\begin{sloppypar}
Die primäre Datenbasis bildet das Erhebungsmaterial des \textit{Bayerischen Sprachatlas}. Für die Untersuchung wurde ein größerer, weniger selektiver Ausschnitt des relevanten Datenmaterials analysiert, als dies in den publizierten BSA-Bän\-den möglich und machbar war. Der Schwerpunkt der vorliegenden Untersuchung wird zwar bedingt durch das zur Verfügung stehende Datenmaterial auf der Flexion des Substantivs liegen, doch ermöglichen die BSA-Daten auch einige Aussagen über die Flexion in der Nominalphrase und den Vergleich formaler Kodierung in verschiedenen Äußerungskontexten anhand von Fallbeispielen.
\end{sloppypar}

\begin{sloppypar}
Eine Untersuchung der dialektalen Flexionsmorphologie eröffnet zugleich the\-o\-re\-ti\-sche Fragestellungen, die über die Dialektologie hinausgehen und eine empirische Fundierung morphologischer Theoriebildung ermöglichen. Dass die Berücksichtigung dialektaler Daten dabei auch eine Herausforderung für die Vorhersagen morphologischer Theorien sein kann, zeigen die oben gegeben Beispiele der oobd. Flexionssysteme. So prognostiziert etwa die Natürliche Morphologie, dass morphologische Systeme durch sogenannte Natürlichkeitsprinzipien bedingt sind und dass diese die Entwicklungsrichtung von morphologischem Wandel festlegen (vgl. \citealt{Mayerthaler1981, Wurzel1984}). Zu diesen Prinzipien gehören das Transparenzprinzip (jeder Form entspricht genau eine Funktion) und das Uniformitätsprinzip (jeder Funktion entspricht genau eine Form). Das Prinzip „one function, one form“ sagt damit einen Abbau von Allomorphien (im Sinne der Uniformität) und von Portmanteau-Morphemen (im Sinne der Transparenz) vorher. Daneben ist das Prinzip des konstruktionellen Ikonismus „eines der konstitutiven Züge jeglicher Morphologieorganisation“ \citep[23]{Mayerthaler1981} und kann unter der Formel zusammengefasst werden, dass ein semantisches „Mehr“ auch durch ein formales „Mehr“ markiert wird. Wie ikonisch eine morphologische Kodierung ist, wird anhand einer Hierarchie beschrieben. Eine Kodierung ist maximal ikonisch, wenn sie additiv und segmentierbar ist (Pl. \teuthoo{hund}{hund}-\teuthoo{n}{n}, \teuthoo{vlaS'n}{vlaʃ̌n}-\teuthoo{A}{α}), sie ist minimal ikonisch, wenn sie nur den Stamm affiziert (\teuthoo{hu2nd}{hūnd} -- \teuthoo{hunt}{hunt}, \teuthoo{hund}{hund} -- \teuthoo{hu?nd}{hünd}) und sie ist nicht-ikonisch wenn das semantische Mehr nicht formal kodiert wird (\teuthoo{hund}{hund} -- \teuthoo{hund}{hund}, \teuthoo{vlas\#n}{vlašn} -- \teuthoo{vlas\#n}{vlašn}). Kontra-ikonisch sind Markierungen, wenn das semantische „Mehr“ durch ein formales „Weniger“, d.\,h. subtraktiv, ausgedrückt wird (hess.-ofr. \teuthoo{ho2nd}{hōnd} -- \teuthoo{ho?n}{hön} ‚Hund‘). Die oobd. Daten zeigen nun, dass durch phonologischen und morphologischen Wandel morphologische Strukturen entstehen, die (1) weniger als maximal-ikonisch und nicht segmentierbar sind und (2) das Prinzip „one function, one form“ unterlaufen wird, da diachron und im Dialektvergleich eine Ausdifferenzierung der formalen Mittel und damit der Allomorphie stattgefunden hat.
\end{sloppypar}

Dass bei Feminina des Typs \teuthoo{vlas\#n}{vlašn} -- \teuthoo{vlas\#n}{vlašn} im Ofr. Nullplurale und damit eine nicht-ikonische Form der Markierung bestehen bleiben, während die fem. Nullplurale im Bair. eher abgebaut werden, ist nicht nur eine Herausforderung der Vorhersagen der Natürlichen Morphologie, sondern auch für \citegen{Bybee1985b} Relevanzprinzip. Nach dieser Theorie korrelieren der semantisch"=funktionale Gehalt einer Kategorie und ihre formale Realisierung relativ zum Stamm: Je stärker eine Kategorie die Bedeutung des Stammes affiziert, desto relevanter ist sie. Je höher nun der Relevanzgrad, so die Vorhersage, desto näher wird die Kategorie formal am Stamm kodiert oder fusioniert mit diesem, und desto stabiler ist sie diachron. Die stammaffizierenden Pluralmarkierungen, die sich im Oobd. finden lassen und die zum Teil auch produktiv sind, weisen Numerus als hochrelevante Kategorie aus. Dass diese hochrelevante Kategorie bei den Feminina formal nicht markiert wird und auch der Definitartikel nicht dis"-am"-bi"-gu"-ie"-rend wirkt, widerspricht daher den Vorhersagen des Relevanzprinzips.

Die untersuchten Daten deuten darauf hin, dass eine Lösung dieses theoretischen Problems darin besteht, Flexion in der morphologischen Theoriebildung grundsätzlich als weniger „statisch“ zu betrachten. In den untersuchten Dialektdaten gibt es Belege dafür, dass flexivische Informationen dann disambiguiert werden, wenn Synkretismen oder zu geringe formale Unterschiede (etwa bei Vokalquantitäts- oder Lenis-Fortis-Kontrasten) das Gelingen der Kommunikation gefährden. Indem die hier kurz vorgestellte morphologische Theoriebildung die Kodierung flexivischer Information durch morphologische Mittel fokussiert, wird sie der Variabilität, die im tatsächlichen Sprachgebrauch besteht, nur bedingt gerecht.\footnote{\textrm{In \citegen{Bybee1985b} Überlegungen zur morphologischen Organisation des Lexikons sind neben flexivischen und lexikalischen Ausdrucksverfahren indes auch syntaktische Verfahren mitangelegt (detaillierter hierzu \sectref{sec:5.2}).}}\largerpage[2]

Insgesamt bestehen der Untersuchungsbereich und die Zielsetzung der Arbeit aus zwei Ebenen, die hier noch einmal zusammengefasst werden. Eine umfassende, systematische Untersuchung der dialektalen Flexionsmorphologie lässt Erkenntnisse zur Entwicklung und Funktion von Deklinationsklassen wie auch zu den nominalen Flexionskategorien Numerus und Kasus und deren einzelnen Kodierungsvarianten in den Dialekten erwarten, steht aber für große Teile des deutschen Dialektgebiets noch aus. Ziel der dialektgeografischen Analyse ist es, die nominale Numerus- und Kasusflexion in den oobd. Dialekten Bayerns für einzelne Tiefenbohrungspunkte in ihrer Systematik kontrastiv darzustellen. Die Darstellung zur Realisierung der Flexionskategorien und zum dialektalen Deklinationsklassensystem ist zunächst primär deskriptiv angelegt. In einem zweiten Schritt werden im Rahmen einer sprachtheoretischen Analyse Fragen zu morphologischem Wandel vor dem Hintergrund morphologischer Theoriebildung behandelt und zur Diskussion gestellt. Die Entwicklung der Dialekte als genuin mündliche Varietäten ist weniger durch kodifizierte Normen geprägt als die schriftsprachlichen Strukturen des Standarddeutschen, weshalb Dialekte eine gute Grundlage zur Überprüfung von Sprachtheorien bilden -- sie sind „Prüfstein“ \citep[368]{Harnisch2000} für vorhandene Theorien.

Die Arbeit ist wie folgt aufgebaut: In \chapref{chap:2} werden zunächst thematische und methodologische Grundlagen der Dialektologie aufgezeigt, an die die vorliegende Arbeit anknüpft und die die Untersuchung als solche motivieren. Teil \ref{part:I} bietet einen Überblick über den aktuellen Forschungsstand zur Nominalflexion, und zwar mit Blick auf die diachrone Entwicklung und auf die typologischen Spezifika des Deutschen. \chapref{chap:3} behandelt hierin Deklinationsklassen als nominales Klassifikationsprinzip, ihren Aufbau und Wandel sowie funktionale Aspekte, in \chapref{chap:4} wird die Realisierung der Flexionskategorien Numerus und Kasus am Substantiv und im syntaktischen Kontext dargestellt. In diesem Zusammenhang werden auch Entwicklungen in den Dialekten des Deutschen in den Blick genommen, doch die eigentliche Darstellung der dialektalen Flexionsmorphologie folgt in Teil \ref{part:II}, dem empirischen Hauptteil der Arbeit. In \chapref{chap:5} werden die behandelten morphologischen Theorien kurz eingeführt. \chapref{chap:6} fasst schließlich das methodische Vorgehen hinsichtlich des untersuchten Dialektraums (\sectref{sec:6.1}), der Materialbasis (\sectref{sec:6.2}) und bei der Datenaufbereitung und -analyse (\sectref{sec:6.3}) zusammen.\largerpage

Im Fokus der Darstellung der Untersuchungsergebnisse in Teil \ref{part:II} steht zunächst das Substantiv. In \chapref{chap:7} erfolgt die dialektgeografische Analyse der Formenbildung in Numerus- und Kasuskategorie im intra- und interdialektalen Vergleich, während in \chapref{chap:8} das dialektale Deklinationsklassensystem in seinem synchronen Aufbau und kontrastiv in der diachronen Entwicklung erläutert wird. In \chapref{chap:9} wird dann der Untersuchungsgegenstand auf die syntaktische Einheit der Nominalphrase erweitert. Gleichzeitig wird anhand einzelner Äußerungskontexte gezeigt, dass die formale Markierung (oder Nicht-Markierung) flexivischer Information in den untersuchten oobd. Dialekten vom semantisch-pragmatischen Kontext abhängig ist. Die Ergebnisse der Analysen bilden die Grundlage von \chapref{chap:10}, das eine Diskussion der dialektalen Flexionsmorphologie vor dem Hintergrund morphologischer Theoriebildung bietet. \chapref{chap:11} fasst schließlich die wichtigsten Ergebnisse der Arbeit zusammen und bietet einen Ausblick auf weitere Forschungsfragen.

\chapter{Dialektologischer Rahmen und Motivation der Arbeit}
\label{chap:2}
Untersuchungen zur dialektalen Flexionsmorphologie werden oft durch die Fest"-stel"-lung eingeleitet und motiviert, dass die Morphologie bisher „keinen Schwerpunkt der Dialektlinguistik“ \citep[44]{Nübling2005} bildete, dass sie gar ein „‚Stiefkind der Dialektologie‘“ \citep[1]{Rowley1997} ist, und auch jüngst wurde der Morphologie im Vergleich zu Phonologie und Syntax „Nachholbedarf“ (\citealt[35]{SchmidtEtAl2019}) attestiert. Im bayerischen Untersuchungsgebiet ist die Forschungssituation vergleichsweise gut. Man muss nicht, wie einst \citet[1]{Rowley1997}, auf \citegen{Schmeller1821} \textit{Die Mundarten Bayerns grammatisch dargestellt} verweisen, um eine wissenschaftliche Darstellung der Morphologie des bair. Raumes zu finden. Durch die Publikation der Morphologie-Bände des \textit{Bayerischen Sprachatlas} sind ausgewählte Phänomene der nominalen Formenbildung in ihrer arealen Dimension für die Dialekte Bayerns zugänglich. \citegen{Rowley1997} Monografie zu den Dialekten Nordostbayerns zeigt zudem exemplarisch, dass eine synchron-deskriptive Formengeografie eines größeren Dialektraums auch die diachrone Dimension berücksichtigen und die Anbindung an morphologische Theoriebildung suchen sollte. Dass morphologische Theorien eine diagnostische und prognostische Dimension haben („morphological theorizing has a diagnostic as well as a prognostic facet“, \citealt[2]{SchallertDammel2019}), zeigen auch (inter alia) die Analysen von \citet{Birkenes2014}, \citet{Harnisch1987} oder \citet{Seiler2003} zu einzelnen dialektalen Phänomenen oder zu ganzen phonologischen und morphologischen Systemen. Die Hinzuziehung und Anwendung dialektaler Daten dient dabei nicht nur der Validierung morphologischer Theorien, sondern kann auch die Erklärung einzelner dialektaler Phänomene leisten. Gleichzeitig sind es, wie \citet[16]{Seiler2003} formuliert, „die theoretischen Fragestellungen, die überhaupt erst die Relevanz dieser Phänomene definieren.“

Die vorliegende Arbeit knüpft an diesen Forschungsansatz an. Grundlegend ist dabei die Auffassung, dass eine dialektale Flexionsmorphologie drei Ebenen berücksichtigen muss: die synchrone Ebene der formalen Realisierung, und zwar mit Blick auf das Gesamtsystem eines Dialekts, die diachrone Dimension des phonologischen und morphologischen Wandels und schließlich eine theoretische Ebene, die wahlweise die synchrone oder die diachrone Dimension berücksichtigt (vgl. \citealt[383]{Harnisch2000}). Zentral ist m.\,E. in der synchronen Dimension des Sprachsystems die Frage nach der mentalen Repräsentation von Flexion, wie es auch \citet[175]{HarnischRowley1990} vor dem Hintergrund einer „Natürlichen Dialektologie“ formulieren: „Sprachwissenschaft synchron und unter dem Blickwinkel des Systemcharakters von Sprache betreiben, heißt mithin auch immer, sprachliche Erscheinungen erklärend auf die Sprecherpsyche beziehen.“ Mit Blick auf die diachrone Dimension und möglicherweise dialektspezifischen Sys\-temwandel lautet die Leitfrage: Welche (phonologischen und morphologischen) Voraussetzungen führen zu welchen „Ergebnissen“ in den rezenten Dialekten? Die „Ergebnisse“ beziehen sich dabei auf die rezenten morphologischen Strukturen, d.\,h. die formalen Kodierungstypen, und auf den Grad ihrer Funktionalisierung in den einzelnen Dialekten.

Um Forschungsfragen wie diese zu beantworten, wird v.\,a. in der jüngeren Dialektologie immer wieder auf die beinahe programmatisch gebrauchte Metapher des Dialektlabors verwiesen (z.\,B. \citealt{deVogelaerSeiler2012,HarnischRowley1990,SchmidtHerrgen2011}). Dialekte stellen demnach einen Zugangspunkt zur Erforschung sprachdynamischer Prozesse dar, um Hypothesen zum Sprachwandel oder zum Sprachsystem zu erproben. Die Idee eines Dialektlabors ist dabei nicht neu, bereits \citet{Moulton1962} verwendet dieses Bild. Zwar sei die Sprachwissenschaft keine experimentelle Wissenschaft, d.\,h. es gibt in dem Sinne keine kontrollierte Versuchsanordnung, doch sind Sprachdaten „so rich and varied that they provide us with an almost limitless number of permutations and combinations to work with“ \citep[460]{Moulton1962}. Dialektale Daten sind dabei „homogeneous enough to be cohesive, but also heterogeneous enough to be interesting and revealing“ \citep[460--461]{Moulton1962}. Hier ist die Idee, typologisch ähnliche und historisch eng verwandte Varietäten zu kontrastieren und über die „mikroskopische Variation“ \citep[185]{Seiler2008} in der arealen Dimension Aussagen über Sprachwandel zu treffen (vgl. \citealt[13]{Bisang2004}, \citealt[5--6]{SchallertDammel2019}, \citealt[1--2]{deVogelaerSeiler2012}, \citealt[816--817]{Rabanus2010}, \citealt[38]{SchmidtEtAl2019}).

Im Rahmen dieses dialektvergleichenden Forschungsansatzes waren insbesondere die methodischen Überlegungen der strukturalistischen (auch strukturellen) Dialektologie maßgebend, die \citet{Weinreich1954} mit dem Aufsatz \textit{Is a structural dialectology possible?} einleitete (einen Überblick zur Forschungsdiskussion bietet \citealt{Gordon2018}, vgl. \citealt{Barbiers2010}, \citealt{Jongen1982}). Der Ansatz der strukturellen Dialektologie führt ein strukturalistisches Verständnis von Sprache als System und die Analyse sprachinterner Relationen mit der sprachgeografischen Ausrichtung der sogenannten traditionellen Dialektologie zusammen. Phonologie und Morphologie werden auch in der traditionellen, junggrammatisch geprägten Dialektologie im Kontext des Gesamtsystems beschrieben, den Referenzpunkt bilden dabei frühere Sprachstufen des Deutschen (vgl. \citealt{Rowley1997}: 18 sowie -- als Überblick -- \citealt{Murray2010} und \citealt{Schrambke2010}). In der strukturellen Dialektologie besteht das Bezugssystem indes nicht in der Diachronie, sondern in einem synchronen Diasystem: „A specifically structural dialectology would look for the structural consequences of partial differences within a framework of partial similarity“ (\citealt[390]{Weinreich1954}, vgl. \citealt[453]{Moulton1962}). Ziel einer strukturellen Dialektologie ist es, „Strukturübereinstimmungen und -unterschiede zwischen ursprungsverwandten und im Raum aneinander grenzenden Mundartsystem aufzudecken, darzustellen und zu deuten“ \citep[248]{Jongen1982}, und besteht damit v.\,a. aus „System- und Strukturvergleichen“ (ebd.: 250).

Vor dem Hintergrund dieses Forschungsziels verweist \citet{Pilch1972} auf methodische Schwierigkeiten in der Praxis der Datenerhebung. Um strukturelle Daten zu erheben und ein entsprechendes Fragebuch zu erstellen, brauche es ein detailliertes Vorwissen über Strukturen und Relationen im System der untersuchten Dialekte; idealiter werde das Fragebuchmaterial durch spontansprachliche Daten ergänzt, um die direkt erhobenen Daten zu verifizieren (vgl. \citealt[180]{Pilch1972}). Anders formuliert: Distinkte Einheiten und relevante Strukturen können nur dann systematisch erhoben werden, wenn (1) vorab bekannt ist, dass es sie gibt und (2) das Fragebuch geeignete Items enthält, die auch dialektübergreifend funktionieren. Im Umkehrschluss heißt das, dass Phänomene, die vorab nicht als relevant bekannt sind, auch nicht systematisch erhoben werden. Eine weitere Schwierigkeit der Datenanalyse, die auch in der vorliegenden Untersuchung immer wieder als solche thematisiert werden wird, ist die Ebene der Transkription. Strukturelle Zusammenhänge werden auf Basis phonetischer Transkriptionen analysiert, doch sind diese phonetischen Transkriptionen unter Umständen „ambiguous in phonemic terms“ \citep[81]{Gordon2018}. Exemplarisch sei hier auf die Lenis-Fortis-Kontraste im Bair. verwiesen, die ein morphophonologischer Marker der Pluralinformation sind, nicht immer aber eindeutig als Lenis- oder Fortisobstruenten realisiert (respektive transkribiert) werden (vgl. \sectref{sec:7.1.2.3.1}).

\begin{sloppypar}
Die vorliegende Untersuchung knüpft sowohl an die junggrammatische als auch an die strukturalistische Forschungstradition an, indem primär Dialektdaten aus dem \textit{Bayerischen Sprachatlas} und daneben aus junggrammatisch und strukturalistisch geprägten Dialektgrammatiken verwendet werden (vgl. \sectref{sec:6.2}). Beide Datentypen stellen einen guten Ausgangspunkt dar, um Forschungsfragen zur nominalen Formenbildung zu bearbeiten, auch wenn spezifischere Aspekte zu Frequenz oder Produktivität nur bedingt beantwortet werden können (vgl. \citealt[391--392]{NickelKürschner2019}). Forschungsfragen, die syntagmatische Relationen oder die Realisierung flexivischer Information in diversen semantisch-pragmatischen Kontexten betreffen, können mit dieser Datenbasis nur anhand von jenen Fallbeispielen behandelt werden, die abgefragt oder, im Falle von Dialektgrammatiken, erwähnt wurden. Für eine systematische und umfängliche Analyse bräuchte es gesprächsbasierte, spontansprachliche Korpora, die für eine dialektvergleichende morphologische Untersuchung zugänglich und entsprechend digital aufbereitet sind (vgl. \citealt[39]{SchmidtEtAl2019}).\footnote{Neben den direkt erhobenen und transkribierten Daten der einzelnen BSA-Projekte gibt es spontansprachliche Audiodaten, die für eine flexionsmorphologische Analyse noch aufbereitet und digital durchsuchbar werden müssten.} Das eingangs aufgeführte Desiderat zur Morphologie betrifft -- zumindest für das bayerische Untersuchungsgebiet -- damit weniger die Forschungslage als solche, sondern vielmehr die Datenlage.
\end{sloppypar}
