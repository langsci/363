\addchap{\lsAbbreviationsTitle}
% \addchap{Abbreviations and symbols}

\section*{Allgemeine Abkürzungen}
\begin{multicols}{2}
\begin{tabbing}
mecklenburg. \= Internationales Phonetisches\kill
ahd. \>   althochdeutsch\\
alem. \>   alemannisch\\
Akk. \>   Akkusativ\\
bair. \>   bairisch\\
brandenb. \>  brandenburgisch\\
Dat. \>   Dativ\\
Dim. \>   Diminutiv\\
DK \>   Deklinationsklasse\\
fem. \>    feminin\\
frnhd. \>   frühneuhochdeutsch\\
Gen.  \>  Genitiv\\
germ. \>   germanisch\\
hd. \>   hochdeutsch\\
hess. \>   hessisch\\
idg. \>   indogermanisch\\
IPA \>   Internationales \\ \> Phonetisches \\ \> Alphabet\\
mask. \>    maskulin\\
mecklenburg. \> mecklenburgisch\\
md.  \>  mitteldeutsch\\
mhd. \>    mittelhochdeutsch\\
mittelfr. \> mittelfränkisch\\
nd. \>   niederdeutsch\\
neutr. \>    neutrum\\
nhd. \>    neuhochdeutsch\\
Nom. \>   Nominativ\\
NP  \>   Nominalphrase\\
obd. \>   oberdeutsch\\
ofr. \>    ostfränkisch\\
omd. \>    ostmitteldeutsch\\
oobd. \>   ostoberdeutsch\\
Pl. \>   Plural\\
Ps. \>   Person\\
ripuar. \>   ripuarisch\\
schwäb. \> schwäbisch\\
Sg. \>   Singular\\
thüring. \>  thüringisch\\
UG  \>  Untersuchungsgebiet\\
UL  \>   Umlaut\\
westfäl. \> westfälisch
\end{tabbing}
\end{multicols}

\section*{Abkürzungen für Sprachatlanten und Sprachatlasprojekte}
\begin{multicols}{2}
\begin{tabbing}
SNOB \= \textit{Sprachatlas von Nordostbayern}\kill
BSA  \>  \textit{Bayerischer Sprachatlas}\\
SBS \>   \textit{Sprachatlas von Bayerisch-} \\ \> \textit{Schwaben}\\
SMF \>   \textit{Sprachatlas von Mittelfranken}\\
SNiB \>   \textit{Sprachatlas von Niederbayern}\\
SNOB \>   \textit{Sprachatlas von Nordostbayern}\\
SOB  \>  \textit{Sprachatlas von Oberbayern}\\
SUF \>   \textit{Sprachatlas von Unterfranken}\\
WA \>   Wenker-Atlas (\textit{Sprachatlas} \\ \> \textit{des Deutschen Reichs})
\end{tabbing}
\end{multicols}


\addchap{Übersicht phonetischer Transkriptionen}

Der Tradition oberdeutscher Sprachatlanten, Dialektwörterbücher und -mo\-no\-gra\-fi\-en folgend wird in dieser Untersuchung das Teuthonista-System als primäres Transkriptionssystem verwendet. Die primäre Datenquelle der Arbeit, die \textit{Bayerische Dialektdatenbank BayDat}, verwendet die Schriftart \textsc{TeuthoBD}. In dieser Monografie wird stattdessen eine Unicode-Darstellung verwendet, die u.\,a. den Vorteil hat, dass die Dialekt-Belege sinnerhaltend in die Zwischenablage kopiert werden können.

\begin{sloppypar}
Die beiden folgenden Tabellen stellen die Transkription von Vokalen und Konsonanten im Teuthonista-System und im Internationalen Phonetischen Alphabet (IPA) zur Übersicht gegenüber. Bei phonetischen Transkriptionen wird im Folgenden aus Gründen der Lesbarkeit auf die entsprechenden Klammern verzichtet. Phonologische Transkriptionen werden durch Schrägstriche /\ldots/ markiert. 
\end{sloppypar}

\begin{table}
\small\tabcolsep=.5\tabcolsep
\begin{tabular}{llllllllllllllllllrc}
\lsptoprule
& \multicolumn{2}{c}{bilabial} & \multicolumn{2}{c}{labio-} & \multicolumn{2}{c}{dental} & \multicolumn{3}{c}{alveolar} & {post-}    & \multicolumn{3}{c}{palatal} & \multicolumn{3}{c}{velar} & \multicolumn{2}{c}{uvular} & glottal\\
& \multicolumn{2}{c}{}         & \multicolumn{2}{c}{dental} & \multicolumn{2}{c}{}       & \multicolumn{3}{c}{}         & {alveolar} & \multicolumn{3}{c}{} & \multicolumn{3}{c}{} & \multicolumn{2}{c}{} & \\
\midrule
 Plosive & \cellcolor{lsLightGray}p & \cellcolor{lsLightGray}b & \multicolumn{2}{c}{} & \cellcolor{lsLightGray}t & \cellcolor{lsLightGray}d & \multicolumn{3}{c}{} &  & \multicolumn{3}{c}{} & \multicolumn{2}{c}{\cellcolor{lsLightGray}k} & \cellcolor{lsLightGray}g & \multicolumn{2}{c}{} & \cellcolor{lsLightGray}\teuthoo{q}{ʔ}\\
         & p & b &  &  & t & d &  &  &  &  &  &  &  & \multicolumn{2}{c}{k} & g &  &  & ʔ\\
 Frikative &  & \cellcolor{lsLightGray}\makecell[tl]{β,\\w\footnote{Die Notation des bilabialen Frikativs variiert zum Teil zwischen [\textrm{W}] und [\textrm{w}], etwa in den Teilprojekten des \textit{Bayerischen Sprachatlas}.}} & \cellcolor{lsLightGray}f & \cellcolor{lsLightGray}v & \cellcolor{lsLightGray}{\teuthoo{S}{ʃ}} & \cellcolor{lsLightGray}s & \cellcolor{lsLightGray}{\teuthoo{S'}{ʃ̌}} & \cellcolor{lsLightGray}{\teuthoo{s\#}{š}} &  &  & \cellcolor{lsLightGray}{\teuthoo{c1}{X\⚬}} & \cellcolor{lsLightGray}{\teuthoo{c}{X}} &  & \cellcolor{lsLightGray}{\teuthoo{x1}{x\⚬}} & \cellcolor{lsLightGray}{\teuthoo{x}{x}} &  & \multicolumn{2}{c}{} & \cellcolor{lsLightGray}h\\
           &  & β & f & v & s & z & \multicolumn{2}{c}{ʃ} &  &  & \multicolumn{2}{c}{ç} &  & \multicolumn{2}{c}{x} &  & \multicolumn{2}{c}{} & h\\
 Nasale & \multicolumn{2}{c}{\cellcolor{lsLightGray}{m}} & \multicolumn{2}{c}{} & \multicolumn{2}{c}{\cellcolor{lsLightGray}{n}} & \multicolumn{3}{c}{} &  & \multicolumn{3}{c}{} & \multicolumn{3}{c}{\cellcolor{lsLightGray}{\teuthoo{N}{ŋ}}} & \multicolumn{2}{c}{} & \\
        & \multicolumn{2}{c}{m} &  &  & \multicolumn{2}{c}{n} &  &  &  &  &  &  &  & \multicolumn{3}{c}{ŋ} &  &  & \\
 Vibrant & \multicolumn{2}{c}{} & \multicolumn{2}{c}{} &  & \cellcolor{lsLightGray}r & \multicolumn{3}{c}{} &  & \multicolumn{3}{c}{} & \multicolumn{3}{c}{} & \multicolumn{1}{c}{} & \cellcolor{lsLightGray}R & \\
         &  &  &  &  &  & r &  &  &  &  &  &  &  &  &  &  & \multicolumn{1}{c}{} & R & \\
 Liquid & \multicolumn{2}{c}{} & \multicolumn{2}{c}{} &  & \cellcolor{lsLightGray}l & \multicolumn{3}{c}{} &  & \multicolumn{3}{c}{} & \multicolumn{3}{c}{} & \multicolumn{2}{c}{} & \\
        &  &  &  &  &  & l &  &  &  &  &  &  &  &  &  &  &  &  & \\
\lspbottomrule
\end{tabular}
\caption{Transkription der Konsonanten im Teuthonista-System (grau schattiert, Fortis/Lenis) versus IPA-System (stimmlos/stimmhaft)\label{tab:1}}
\end{table}

Die Grundzeichen von Plosiven, Frikativen und auch Affrikaten (/p b/, /f v/, /pf bv/ usw.) sind im Teuthonista-System -- anders als im IPA -- grundsätzlich stimmlos. Gegenübergestellt werden im Teuthonista-System jeweils Fortes vs. Lenes, während beim IPA die Opposition stimmlos vs. stimmhaft in  \tabref{tab:1} gegenübergestellt wird. Stimmhaftigkeit wird im Teuthonista-System durch ein eigenes Diakritikum, einen untergesetzten Punkt [\teuthoo{b4}{ḅ} \teuthoo{p4}{p̣}], kodiert (vgl. \tabref{tab:3}).

Bei den Vokalen wird der Öffnungsgrad im Teuthonista-System durch Diakritika markiert: Untergesetzter Punkt (bzw. gedoppelter Punkt) markiert eine (stärker) geschlossene Realisierung, untergesetzter Haken (bzw. gedoppelter Haken) eine (stärker) offene Realisierung, sodass für jeden Grundzeichen-Vokal ein Kontinuum transkribiert werden kann (z.\,B. \teuthoo{i\$}{i̤} – \teuthoo{i.}{iͅ} – \teuthoo{i}{i} – \teuthoo{i.}{iͅ} – \teuthoo{i.}{iͅ}).

\begin{table}
\begin{tabular}{lllllcr}
\lsptoprule
& vorne & & & & zentral & hinten\\
\midrule
\rowcolor{lsLightGray}\cellcolor{white}geschlossen & \teuthoo{i}{i} \teuthoo{u?}{ü} & & & &  &  u\\
& i y & & & & &   u\\
\rowcolor{lsLightGray}\cellcolor{white}halb-geschlossen & & \teuthoo{e}{e} \teuthoo{o.}{ö} & & & \teuthoo{E}{ə} &  \teuthoo{o}{o}\\
& & e ø & & & ə &  o\\
\rowcolor{lsLightGray}\cellcolor{white}halb-offen & & & \teuthoo{e.}{eͅ} \teuthoo{o?.}{öͅ} & & \teuthoo{A}{α} &  \teuthoo{o.}{oͅ}\\
& & & ɛ œ & & ɐ &  ɔ\\
\rowcolor{lsLightGray}\cellcolor{white}offen & & & & \teuthoo{a4}{ạ} & \teuthoo{a}{a} &  \teuthoo{a.}{aͅ} \teuthoo{å}{{\burgeroalpha}}\\
& & & & a &  &  ɑ ɒ\\
\lspbottomrule
\end{tabular}
\caption{Transkription der Vokale (ungerundet, gerundet) im Teuthonista-System (grau schattiert) versus IPA-System}
\label{tab:2}
\end{table}

\begin{sloppypar}
Grundsätzlich funktioniert die Teuthonista-Notation nach dem Bau\-kas\-ten-Prin\-zip: Die phonetische Spezifizierung der Grundzeichen erfolgt durch Diakritika, die bei Bedarf kombiniert werden.  \tabref{tab:3} bietet eine Übersicht der gebräuchlichsten Diakritika. Für eine vollständige Auflistung sämtlicher Diakritika und möglicher Kombinationen in den Transkriptionen des \textit{Bayerischen Sprachatlas} sei an dieser Stelle auf das „Transkriptionsformular“ im \textit{Sprachatlas Bayerisch Schwaben} verwiesen (\citealt[162--165]{SBS1}).
\end{sloppypar}

\begin{table}
\begin{tabularx}{\textwidth}{Qll}
\lsptoprule
& Teuthonista & IPA\\
\midrule
\multicolumn{3}{l}{Lenisierung}\\
\hspace{1ex}leicht lenisierte Fortes & \teuthoo{t,}{t͓} & \\
\hspace{1ex}stark lenisierte Fortes & \teuthoo{t;}{t͓͓} & \\
\multicolumn{3}{l}{Fortisierung}\\
\hspace{1ex}leicht fortisierte Lenes & \teuthoo{d5}{d̩} & \\
\hspace{1ex}stark fortisierte Lenes & \teuthoo{d\%}{d͈} & \\
Aspiration & ph \teuthoo{p\_}{pʰ} & pʰ\\
Stimmhaftigkeit & \teuthoo{b4}{ḅ} \teuthoo{p4}{p̣} & p̬\\
Stimmlosigkeit & \makecell[tl]{bei\\stimmlosen\\Nasalen\\\teuthoo{m4}{ṃ} \teuthoo{n4}{ṇ} \teuthoo{N4}{ŋ̣}} & m̥ n̥ ŋ̥\\
Silbizität & \teuthoo{n@}{n̥} & n̩\\
\multicolumn{3}{l}{Quantität}\\
\hspace{1ex}Länge & \teuthoo{a2}{ā} \teuthoo{m2}{m̄} & aː mː\\
\hspace{1ex}Halblänge & \teuthoo{a>}{â} & aˑ\\
\hspace{1ex}Kürze & \teuthoo{a3}{ă} & ă\\
\multicolumn{3}{l}{Nasalierung}\\
\hspace{1ex}Nasalierung & \teuthoo{a+}{ã} & ã\\
\hspace{1ex}starke Nasalierung & \teuthoo{a*}{ã} & \\
\multicolumn{3}{l}{Rundung}\\
\hspace{1ex}leichte Rundung & \teuthoo{I?}{ı̈} & i̜\\
\hspace{1ex}stärkere Rundung & \teuthoo{I\textbackslash}{\"{\"{i}}} & i̹\\
Reduktion (von Vokalen, Aspiration) & z.B. \textsuperscript{i} \teuthoo{_}{ʰ} & \\
\hspace{1ex}Reduktion des zweiten Elements bei Diphthongen & a\textsuperscript{i} & ai̯\\
\multicolumn{3}{l}{Zwischenwertnotationen, z.\,B.}\\
\hspace{1ex}Grenzwert Schwa\slash Tiefschwa & \teuthoo{Å}{{\burgershwaalpha}} & \\
\hspace{1ex}Grenzwert Plosiv\slash Frikativ (Spirantisierung von /b/) & \teuthoo{æ}{{\burgerbw}} & \\
Klammerung von Diakritika zur Notation von Annäherungswerten & \makecell[tl]{z.B.\\\teuthoo{a94}{a\klammeruntenpost{}̣} \teuthoo{a(+}{ã\klammerobenpost{}}} & \\
\lspbottomrule
\end{tabularx}
\caption{Diakritika im Teuthonista- versus IPA-System}
\label{tab:3}
\end{table}
