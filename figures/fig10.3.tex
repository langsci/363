\documentclass{standalone}
\usepackage{tikz}
\usetikzlibrary{matrix}
\usetikzlibrary{positioning}
\usepackage{libertinus}
\usepackage{todonotes}
\begin{document}
	\begin{tikzpicture}\footnotesize
		\matrix (schema)
		[matrix of nodes, 
		nodes in empty cells,
		nodes={text width=2.8cm, align=center},
		row sep=.33cm
		]
		{
			& & di dašn ,die Tasche` & & \\[.3cm]
			& & di dašn ‚die Taschen` & & \\
			& dα haufn ,Der Haufen` & di haufn ,die Haufen` & & \\
			& əs vensdər ,das Fenster` & & di vensdər ,die Fenster` & \\
			\textbf{Singular} & & & & \textbf{Plural} \\ 
			& & & & \\[0.5ex]
			& \textbf{mehrsilbig} & \textbf{mehrsilbig} & \textbf{mehrsilbig} & \textbf{mehrsilbig}\\[0.1pt]
			& \textbf{Reduktionssilbe mhd. -\textit{en}, -\textit{er}} & \textbf{Reduktionssilbe mhd. -\textit{en}} & \textbf{Reduktionssilbe mhd. -\textit{er}} & \textbf{Reduktionssilbe bair. -\textit{αn} oder Nasal+\textit{α}}\\
			& \textbf{\textit{der/das}} & \textbf{\textit{die}} & \textbf{\textit{die}} & \textbf{\textit{die}} \\
			& svensdα ,das Fenster` & & bvensdα ,die Fenster` &\\
		& dα haufα ,der Haufen` & di haufα ,die Haufen` & & di haufαn ,die Haufen` \\
		& & di biəkα ,die Birke` & & di biəkαn ,die Birken` \\
		& & di dašn ,die Tasche` & & di dašnα ,die Taschen` \\
		};
		\node[left = 0.1cm of schema-3-1] {\textbf{Ofr.}};
		\node[left = 0.0cm of schema-11-1] {\textbf{Bair.}};
		%draws für linien in oberer gruppe
		\draw [dotted] (schema-1-3)  -- (schema-2-3);
		\draw [dotted] (schema-3-2.center) ++(1.35cm,0pt) -- (schema-3-3.west) -- ++(0.16cm,0pt);
		\draw [anchor=east, dotted] (schema-4-2) -- (schema-4-4);
		\draw [left color=black, %draw für die zweite gruppe
		right color=black,
		middle color=white]
		(schema-6-1.north west) rectangle (schema-6-5.south east);
		%draws für die dritte gruppe
		\draw [dotted] (schema-10-2)  -- (schema-10-4);
		\draw [dotted] (schema-11-2.center) ++(1.3cm,0pt) -- (schema-11-3.west) -- ++(0.2cm,0pt);
		\draw [dotted] (schema-11-3) -- (schema-11-5);
		\draw [dotted] (schema-12-3) -- (schema-12-5);
		\draw [dotted] (schema-13-3) -- (schema-13-5);
	\end{tikzpicture}
\end{document}
