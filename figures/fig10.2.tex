\documentclass{standalone}
\usepackage{tikz}
\usetikzlibrary{matrix}
\usetikzlibrary{positioning}
\usepackage{fontspec}
\setmainfont
[
Ligatures={TeX,Common},
PunctuationSpace=0,
Numbers={Proportional},
BoldFont = LibertinusSerif-Semibold.otf,
ItalicFont = LibertinusSerif-Italic.otf,
BoldItalicFont = LibertinusSerif-SemiboldItalic.otf,
BoldSlantedFont = LibertinusSerif-Semibold.otf,
SlantedFont    = LibertinusSerif-Regular.otf,
SlantedFeatures = {FakeSlant=0.25},
BoldSlantedFeatures = {FakeSlant=0.25},
SmallCapsFeatures = {FakeSlant=0},
]
{LibertinusSerif-Regular.otf} 
\usepackage{todonotes}
\begin{document}
	\begin{tikzpicture}\footnotesize
		\matrix (continuum)
		[matrix of nodes, 
		nodes in empty cells,
		nodes={text width=2.8cm, align=center},
		row sep=.2cm
		]
		{
			\textbf{Singular} & & & & \textbf{Plural} \\ 
			& & & & \\[0.5ex]
			\textbf{einsilbig} & \textbf{mehrsilbig} & \textbf{mehrsilbig} & \textbf{einsilbig} & \textbf{mehrsilbig}\\[0.1pt]
			\textbf{Langvokal} & \textbf{Reduktionssilbe \textit{-n, -α}} & \textbf{Reduktionssilbe \textit{-n, -α}} & \textbf{Kurzvokal} & \textbf{Reduktionssilbe \textit{-αn} oder Nasal+\textit{α}}\\
			\textbf{Konsonantenelision im Auslaut oder Lenisobstruent} & & & \textbf{Konsonant im Auslaut oder Fortisobstruent} &  \\
			\textbf{\textit{der/das}} & \textbf{\textit{der/das}} & \textbf{die} & \textbf{\textit{die}} & \textbf{\textit{die}} \\
			dα vīš ,der Fisch` & & & bviʃ̌ ,die Fische` & \\
			dα bō ,der Bach` & & & di bạx ,die Bäche` & \\
			snēsd ,das Nest` & & di neʃtα ,die Nester` & & \\ 
			& & di dašn ,die Tasche` & & di dašnα ,die Taschen` \\
			& & di biəkα ,die Birke` & & di biəkαn ,die Birken` \\
			& dα haufα ,der Haufen` & di haufα ,die Haufen` & & di haufαn ,die Haufen` \\
		};
		\draw [left color=black, 
		right color=black,
		middle color=white]
		(continuum-2-1.north west) rectangle (continuum-2-5.south east);
		\draw [dotted] (continuum-7-1)  -- (continuum-7-4);
		\draw [dotted] (continuum-8-1)  -- (continuum-8-4);
		\draw [dotted] (continuum-9-1)  -- (continuum-9-3);
		\draw [dotted] (continuum-12-2.center) ++(1.3cm,0pt) -- (continuum-12-3.west) -- ++(0.2cm,0pt);
		\draw [dotted] (continuum-10-3) -- (continuum-10-5);
		\draw [dotted] (continuum-11-3) -- (continuum-11-5);
		\draw [dotted] (continuum-12-3) -- (continuum-12-5);
	\end{tikzpicture}
\end{document}

ʃ
š
ī
ō
ē
ǝ
̆
ạ
̌ʃ
̌
